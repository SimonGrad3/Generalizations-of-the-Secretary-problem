\documentclass[a4paper,12pt]{article}

\usepackage[slovene]{babel}
\usepackage{amsfonts,amssymb,amsmath}
\usepackage[utf8]{inputenc}
\usepackage[T1]{fontenc}
\usepackage{lmodern}
\usepackage{graphicx}

\def\qed{$\hfill\Box$}   % konec dokaza
\def\qedm{\qquad\Box}   % konec dokaza v matematičnem načinu
\newtheorem{izrek}{Izrek}
\newtheorem{trditev}{Trditev}
\newtheorem{posledica}{Posledica}
\newtheorem{lema}{Lema}
\newtheorem{pripomba}{Pripomba}
\newtheorem{definicija}{Definicija}
\newtheorem{zgled}{Zgled}


\begin{document}

\begin{center}
    \Large
    \textbf{Generalizations of the Secretary problem}
        
    \vspace{0.4cm}
    \textbf{Simon Grad}
    \vspace{0.9cm}
    
\end{center}

    Optimal stopping theory is a branch of decision theory concerned with determining 
    the best time to take a particular action when sequentially encountering options. 
    This theory has applications in various fields, including economics, statistics, 
    computer science, and even everyday decision-making scenarios. 
    One classic problem that exemplifies optimal stopping theory is the Secretary Problem.

    The Secretary Problem, also known as the Marriage Problem, 
    poses the following scenario: 
    Suppose you have a series of candidates interviewing for a position, 
    and you must hire exactly one. The candidates arrive sequentially, 
    and after each interview, you must decide whether to hire the candidate or move on to 
    the next one. However, once you reject a candidate, they cannot be reconsidered. 
    The goal is to maximize the probability of selecting the best candidate.  

    This problem captures the essence of optimal stopping theory because it highlights 
    the tension between the desire to gather more information (interview more candidates) and 
    the need to make a decision within a limited timeframe. 
    The optimal strategy for the Secretary Problem dictates that one should 
    reject the first $\frac{1}{e}$
    (approximately $37\%$) of candidates and then hire the first candidate who is better 
    than all previously interviewed candidates. 
    This strategy is proven to maximize the probability of selecting the best candidate.
    
    Solving the Secretary Problem analytically provides elegant insights into optimal 
    stopping strategies. However, experimental studies offer a complementary approach to 
    validate theoretical findings and explore nuances in decision-making behavior.

    A simple Python program has been developed to simulate an example of partner selection, 
    where there are $n$ partners numbered from 1 to $n$ (with 
    $n$ representing the most suitable partner). Initially, we evaluate the first 
    $m$ partners, and then we choose to marry the next partner who is superior to all 
    previously encountered partners. The program either returns the partner we marry or 
    notifies us if we fail to find a suitable partner.

    The subsequent step involves running this program multiple times with varying 
    stopping points ($m$) for $n = 100$ partners to demonstrate that stopping at 
    $m = 37$ indeed yields the optimal outcome. Extending this analysis to larger values of 
    $n$ can further enhance the accuracy of our results.

    With the same experimental process we will try to answer questions like:
    Is there a distinction when considering individuals of equal suitability, 
    and does this affect our results? 
    Moreover, does our outcome differ depending on whether we strictly prefer a 
    superior candidate or are content with someone ranking equally to candidates 
    we've previously encountered?
    What criteria is best if we want to hire two people and want to maximize the 
    probability of selecting the best two candidates? 
    What criteria is best if we want to hire two people and want to minimize the average 
    rank probability of the selected two candidates? 
    What strategy is best if we can choose two people and we are happy if one of them is the best? 
    What criteria is best if we want to hire k people and want to minimize the rank of the worse 
    one that was hired? Etc.


    I will primarily work with Python, potentially utilizing Jupyter extensions, 
    and will focus on discrete-time case. 









 













\end{document}